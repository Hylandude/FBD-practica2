%%%%%%%%%%%%%%%%%%%%%%%%%%%%%%%%%%%%%%%%%
% fphw Assignment
% LaTeX Template
% Version 1.0 (27/04/2019)
%
% This template originates from:
% https://www.LaTeXTemplates.com
%
% Authors:
% Class by Felipe Portales-Oliva (f.portales.oliva@gmail.com) with template 
% content and modifications by Vel (vel@LaTeXTemplates.com)
%
% Template (this file) License:
% CC BY-NC-SA 3.0 (http://creativecommons.org/licenses/by-nc-sa/3.0/)
%
%%%%%%%%%%%%%%%%%%%%%%%%%%%%%%%%%%%%%%%%%

%----------------------------------------------------------------------------------------
%	PACKAGES AND OTHER DOCUMENT CONFIGURATIONS
%----------------------------------------------------------------------------------------

\documentclass[
	10pt, % Default font size, values between 10pt-12pt are allowed
	%letterpaper, % Uncomment for US letter paper size
	spanish % Uncomment for Spanish
]{fphw}

% Template-specific packages
\usepackage[utf8]{inputenc} % Required for inputting international characters
\usepackage[T1]{fontenc} % Output font encoding for international characters
\usepackage{mathpazo} % Use the Palatino font
\usepackage{graphicx} % Required for including images
\usepackage{booktabs} % Required for better horizontal rules in tables
\usepackage{listings} % Required for insertion of code
\usepackage{enumerate} % To modify the enumerate environment
\usepackage[spanish, es-tabla]{babel}
\usepackage{spverbatim}

%----------------------------------------------------------------------------------------
%	ASSIGNMENT INFORMATION
%----------------------------------------------------------------------------------------

\title{Reporte Práctica \#2} % Assignment title

\author{César Augusto Farrera Ortega (311617670)
Karem Ramos Calpulalpan - 314068583} % Student name

\date{02/03/2020} % Due date

\institute{UNAM \\ Facultad de Ciencias} % Institute or school name

\class{Fundamentos de bases de datos} % Course or class name

\professor{Gerardo Avilés} % Professor or teacher in charge of the assignment

%----------------------------------------------------------------------------------------

\begin{document}

\maketitle % Output the assignment title, created automatically using the information in the custom commands above

%----------------------------------------------------------------------------------------
%	ASSIGNMENT CONTENT
%----------------------------------------------------------------------------------------

\section*{Analizáis de requerimientos}


\subsection*{Enumeración de los requerimientos candidato}
\begin{enumerate}
	\item Requerimiento 1
\end{enumerate}

\subsection*{Comprensión del contexto del sistema}

Diagrama Entidad relacion

\subsection*{Captura de requerimientos funcionales}

\subsection*{Captura de requerimientos no funcionales}

%----------------------------------------------------------------------------------------

\section*{Preguntas}

\begin{problem}
Describe cinco diferencias entre almacenar la información utilizando un sistema de archivos a almacenarla utilizando una base
de datos. Deberás describir que es mas conveniente utilizar
\end{problem}

\subsection*{Respuesta}

\begin{enumerate}
	\item Diferencia 1
\end{enumerate}

%----------------------------------------------------------------------------------------

\begin{thebibliography}{1}
  		\bibitem {}Referencia 1
  		
\end{thebibliography}

\end{document}
